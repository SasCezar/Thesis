\chapter{State of the Art}
In this chapter we will introduce a quick summary of the state-of-the-art techniques in entity linking based on the survey of Shen et al.\cite{shen2015entity}.
\paragraph{}
Shen \cite{shen2015entity} divides the main task in 3 modules:
\begin{itemize}
\item Candidate entity generation: This module extract for each entity mention a set of candidates which may contain the correct one.
\item Candidate entity ranking: Its task is to find the most likely link for the mention in the list of the candidates.
\item NIL prediction: Some mentions could not have a link, this module checks if the best candidate from the previous module is the target for the mention. 
\end{itemize}

\section{Candidates Generation}
Formally, the candidate entity generation can be described as follows:
\[\forall m \in M \; find \; E_m\]
Where:

\begin{itemize}[noitemsep,  topsep=10pt]
\item $m$ is the entity mention
\item $M$ is the set of all mentions
\item $E_m$ is the set of candidate entities for m
\end{itemize}

There are different approaches in literature for the candidates generation, these methods are not based on a single technique, in fact most of them combines two or more main techniques to overcome some problems and achieve better results. We can classify these techniques in three groups:
\begin{itemize}[noitemsep,  topsep=10pt]
\item Name Dictionary Based Techniques
\item Search Engine Based Techniques
\item 
\end{itemize}

\section{Candidates Ranking}
After the candidates extraction, defined in the previous section, we need to rank the candidates.

\section{NIL Prediction}
Some entity mentions might not have a corresponding record in the KB, therefore we have to deal with the problem of predicting unlinkable mentions. 