\chapter{Introduction}
\paragraph{}
With the diffusion of Internet, the Web 2.0, and the mobile devices the amount of data generated grows exponentially. A large amount of this data is user generated content. As shown by \href{http://www.internetlivestats.com/}{www.internetlivestats.com} each day 500 million tweets are sent and 5 million blog posts are written. The problem with these articles and posts is that they are not structured and this means that we cannot use them as information without pre-processing. The field of computer science that tries to generate information from unstructured data is called Information Extraction (IE). In most cases IE concerns of processing human language text using Natural Language Processing (NLP) techniques. 

\section{Information Extraction}
IE can be separated in many sub-tasks based on the data required to process. Typically the main sub-tasks are:

\begin{itemize}[itemsep = 0.1em]
\item Named Entity rEcognition and Linking (NEEL): which is composed by a Named Entity Recognition (NER) system and a Named Entity Linking (NEL) system. A named entity is a real-world object such as persons, locations, organizations, products, etc., that can be denoted with a proper name. The NER goal is to find all the named entity contained in the text. The NEL takes the named entities and links them to the knowleadge base (KB).

\begin{figure}[h!]
\caption{NEEL System}
\label{fig:neel}
\end{figure}

\item Relationship extraction: where the aim is to find the relations between entities in a phrase. More formally, the task of relationship extraction can be defined as a classification problem where given two entities \(e_1\)and \(e_2\) and a phrase \(s\) the classifier returns a label \(r\) describing the relation between the two entities (formally \((e_1, e_2, s)\rightarrow r\)). For example for the sentence ``Elon lives in California'' we can extract ``PERSON \textit{located} in LOCATION".

\item Terminology extraction: which consist in finding the relevant terms in a text or more generally in a corpus. This can be used for creating a domain knowledge based on the terms that describe the domain.
\end{itemize}

In this thesis, we will focus only on the NEEL task (more precisely on the linking part).
\section{Named Entity rEcognition and Linking}
\subsection{Named Entity Recognition}
\paragraph{}
Named Entity Recognition is a critical IE task, as it identifies which terms in a text are mentions of entities in the real world.
As we see in the figure \ref{fig:ner_io}, the NER takes a text (a tweet in our case) as an input and returns the entities with the corresponding type. The NER is also a pre-requisite not only for NEL but also for other IE task, including co-reference resolution, and relation extraction. In the image below we can see an example of a NER running on a tweet. \\

\begin{figure}[ht]
\caption{Example of a NER I/O}
\label{fig:ner_io}
\end{figure}

\paragraph{}
As we mentioned before user content are one of the biggest source of data, specifically platforms of microblogging like Facebook and Twitter. Early experiments have showed this genre to be extremely challenging for state-of-the-art algorithms of IE~\cite{derczynski2013microblog}. For instance, NER methods typically have around 90\% of accuracy on longer texts, but only 35-50\% over microblog post like tweets. The reasons of this difficulty can be summarized as follows:

\begin{itemize}[itemsep = 0.1em]
\item Shortness of microblogs: the max length of a tweet is 140 characters and this makes them hard to interpret.
\item Capitalization of the words: in a tweet, or any other microblog post, the capitalization of words may be ignored for increasing the speed of writing. The user could also deliberately overdo them with the intent of adding more emphasis to the message. This is a problem as some words change their meaning based on the capitalization of the letters. For example the words ``trump" and ``Trump" in a tweet could both be refereed to the president of the USA, but for a musician the first one might refer to a musical instrument and the second one to the president.
\item Word Typos: as for the capitalization a typo could significantly change the meaning of a word. In microblogs posts the amount of typos is 2.5 times greater than in a well-formed text~\cite{derczynski2015analysis}.
\item Abbreviations: given the limit on the number of characters, users tend to use abbreviations in order to write more expressive messages in the same amount of space.
\item Emotions: the meaning of a sentence can be drastically changed by an emoji. 
\end{itemize}

\paragraph{}
To overcome these problems the researchers focused on specific NERs for microblogs posts (eg Named Entity Recognition for Twitter using Conditional Random Fields by Ritter et. al~\cite{ritter2011named}).

\subsection{Named Entity Linking}
\paragraph{}
Named Entity Linking is the task of determining the identity of entities mentioned in the text. Sometimes is also called Named Entity Disambiguation (NED), and also Named Entity Normalization (NEN). It typically requires annotating a potentially ambiguous entity mention with a link to a canonical identifier, the knowledge base, describing the entity. A popular choice for the the knowledge base is Wikipedia, in which each page is considerate a named entity 

\begin{figure}[ht]
\caption{Example of a NEL I/O}
\label{fig:nel_io}
\end{figure}

\paragraph{}
The problems we previously introduced for the NER are also valid for the linking problem (except for the emoticons). Also named entity mention can be highly ambiguous. For example, given the sentence ``Paris is the capital of France" and the entity ``Paris", the idea is to determine that ``Paris" is referred to the city and not to ``Paris Hilton" the American celebrity. Another NEL problem is represented by the fact that the same entity could have multiple surface forms. An example could be the named entity ``USA" also refereed as ``America", ``US", ``United State of America", and ``United States". We also need to keep in mind that some words recognized by the NER might not have a corresponding description in the KB, we refer to these words as Out of Vocabulary (OOV). These problems needs to be addressed by any NEL as they are very common and could decrease the overall performance. One possible way to solve these issues is by considering the context of the named entity. This is a bit more complex for tweets and other microblog posts, so the researchers found different techniques to solve this problem that we will discuss in the next chapter.
